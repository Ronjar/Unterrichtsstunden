\documentclass{article}
\usepackage[utf8]{inputenc}
\usepackage[T1]{fontenc}
\usepackage[ngerman]{babel}
\usepackage{longtable}
\usepackage{booktabs}
\usepackage{listings}
\usepackage{array}
\usepackage{geometry}
\usepackage{hyperref}
\usepackage{parskip}
\geometry{a4paper, margin=2cm}

\usepackage{inconsolata}

\usepackage{helvet}
\renewcommand{\familydefault}{\sfdefault}


\usepackage{color}

% Solarized Farben definieren
\definecolor{solarBase03}{rgb}{0.0, 0.11, 0.16}
\definecolor{solarYellow}{rgb}{0.93, 0.71, 0.0}
\definecolor{solarRed}{rgb}{0.86, 0.26, 0.18}
\definecolor{solarGreen}{rgb}{0.18, 0.55, 0.34}
\definecolor{solarBlue}{rgb}{0.0, 0.39, 0.51}
\definecolor{solarGray}{rgb}{0.58, 0.58, 0.58}

% lstset für Solarized
\lstset{
  language=Java,
  backgroundcolor=\color{solarBase03},
  basicstyle=\ttfamily\color{solarGray},
  keywordstyle=\color{solarBlue}\bfseries,
  commentstyle=\color{solarGreen}\itshape,
  stringstyle=\color{solarRed},
  showspaces=false,
  showtabs=false,
  breaklines=true,
  showstringspaces=false,
  breakatwhitespace=true,
  moredelim=[il][\textcolor{solarGray}]{$$},
  moredelim=[is][\textcolor{solarGray}]{\%\%}{\%\%},
  frame=single,
  rulecolor=\color{solarGray},
  numbers=left,
  numberstyle=\tiny\color{solarGray},
  stepnumber=1,
  numbersep=5pt
}

\begin{document}

\title{Glossar der grundlegenden Processing-Funktionen}
\author{}
\date{}
\maketitle

\section*{Grundlegende Struktur-Funktionen}

\begin{longtable}{|p{0.28\textwidth}|p{0.28\textwidth}|p{0.36\textwidth}|}
\hline
\textbf{Code} & \textbf{Parameter} & \textbf{Erklärung} \\
\hline
\endfirsthead

\hline
\textbf{Code} & \textbf{Parameter} & \textbf{Erklärung} \\
\hline
\endhead

\lstinline|void setup()| & & Wird einmal zu Beginn des Programms ausgeführt. Hier werden Anfangseinstellungen vorgenommen, z.B. \lstinline|size()| oder \lstinline|background()|. \\
\hline

\lstinline|void draw()| & & Wird nach \lstinline|void setup()| fortlaufend ausgeführt. Hier kommt der Hauptcode für Zeichnungen und Animationen hin. \\
\hline

\lstinline|size(int width, int height)| & 
\begin{tabular}[t]{@{}l@{}}
\texttt{int width}: \textit{Breite in Pixeln} \\
\texttt{int height}: \textit{Höhe in Pixeln}
\end{tabular}
& Legt die Größe des Zeichenfensters fest. \\
\hline

\lstinline|frameRate(float fps)| & 
\texttt{float fps}: \textit{Anzahl der Frames pro Sekunde} & Legt die Anzahl der Bilder pro Sekunde fest. \\
\hline


\end{longtable}

\section*{Farb- und Zeichenstil-Funktionen}

\begin{longtable}{|p{0.28\textwidth}|p{0.28\textwidth}|p{0.36\textwidth}|}
\hline
\textbf{Code} & \textbf{Parameter} & \textbf{Erklärung} \\
\hline
\endfirsthead

\hline
\textbf{Code} & \textbf{Parameter} & \textbf{Erklärung} \\
\hline
\endhead

\lstinline|background(int value)| & 
\texttt{int value}: \textit{Grauwert (0–255)} & Setzt die Hintergrundfarbe mit Grauwert. \\
\hline

\lstinline|background(int r, int g, int b)| & 
\begin{tabular}[t]{@{}l@{}}
\texttt{int r}: \textit{Rot (0–255)} \\
\texttt{int g}: \textit{Grün (0–255)} \\
\texttt{int b}: \textit{Blau (0–255)}
\end{tabular}
& Setzt die Hintergrundfarbe mit RGB-Werten. \\
\hline

\lstinline|fill(int value)| & 
\texttt{int value}: \textit{Grauwert (0–255)} & Legt die Füllfarbe für Formen mit Grauwert fest. Gilt, bis eine neue Farbe gesetzt wird.\\
\hline

\lstinline|fill(int r, int g, int b)| & 
\begin{tabular}[t]{@{}l@{}}
\texttt{int r}: \textit{Rot (0–255)} \\
\texttt{int g}: \textit{Grün (0–255)} \\
\texttt{int b}: \textit{Blau (0–255)}
\end{tabular}
& Legt die Füllfarbe für Formen mit RGB-Werten fest. Gilt, bis eine neue Farbe gesetzt wird.\\
\hline

\lstinline|stroke(int value)| & 
\texttt{int value}: \textit{Grauwert (0–255)} & Legt die Linienfarbe mit Grauwert fest. Gilt, bis eine neue Farbe gesetzt wird.\\
\hline

\lstinline|stroke(int r, int g, int b)| & 
\begin{tabular}[t]{@{}l@{}}
\texttt{int r}: \textit{Rot (0–255)} \\
\texttt{int g}: \textit{Grün (0–255)} \\
\texttt{int b}: \textit{Blau (0–255)}
\end{tabular}
& Legt die Linienfarbe mit RGB-Werten fest. Gilt, bis eine neue Farbe gesetzt wird.\\
\hline

\lstinline|strokeWeight(float weight)| & 
\texttt{float weight}: \textit{Linienstärke in Pixeln} & Legt die Dicke der Linien fest. Gilt, bis eine neue Dicke gesetzt wird.\\
\hline

\lstinline|noStroke()| & & Deaktiviert das Zeichnen von Linien (keine Konturen). Gilt, bis ein Stroke gesetzt wird.\\
\hline

\lstinline|noFill()| & & Deaktiviert das Füllen von Formen.  Gilt, bis eine neue Farbe gesetzt wird. \\
\hline


\end{longtable}

% Wiederhole diesen Vorgang für alle weiteren Tabellen...

\section*{Formen zeichnen}

\begin{longtable}{|p{0.28\textwidth}|p{0.28\textwidth}|p{0.36\textwidth}|}
\hline
\textbf{Code} & \textbf{Parameter} & \textbf{Erklärung} \\
\hline
\endfirsthead

\hline
\textbf{Code} & \textbf{Parameter} & \textbf{Erklärung} \\
\hline
\endhead

\lstinline|point(float x, float y)| & 
\begin{tabular}[t]{@{}l@{}}
\texttt{float x}: \textit{X-Position} \\
\texttt{float y}: \textit{Y-Position}
\end{tabular}
& Zeichnet einen Punkt an der Position \texttt{x}, \texttt{y}. \\
\hline

\lstinline|line(float x1, float y1, float x2, float y2)| & 
\begin{tabular}[t]{@{}l@{}}
\texttt{float x1}: \textit{Start-X} \\
\texttt{float y1}: \textit{Start-Y} \\
\texttt{float x2}: \textit{Ende-X} \\
\texttt{float y2}: \textit{Ende-Y}
\end{tabular}
& Zeichnet eine Linie von \texttt{x1}, \texttt{y1} nach \texttt{x2}, \texttt{y2}. \\
\hline

% Füge \hline nach jeder Tabellenzeile hinzu
\lstinline|rect(float x, float y, float w, float h)| & 
\begin{tabular}[t]{@{}l@{}}
\texttt{float x}: \textit{X-Position} \\
\texttt{float y}: \textit{Y-Position} \\
\texttt{float w}: \textit{Breite} \\
\texttt{float h}: \textit{Höhe}
\end{tabular}
& Zeichnet ein Rechteck mit oberer linker Ecke bei \texttt{x}, \texttt{y}, Breite \texttt{w} und Höhe \texttt{h}. \\
\hline

\lstinline|square(float x, float y, float s)| & 
\begin{tabular}[t]{@{}l@{}}
\texttt{float x}: \textit{X-Position} \\
\texttt{float y}: \textit{Y-Position} \\
\texttt{float s}: \textit{Seitenlänge}
\end{tabular}
& Zeichnet ein Quadrat mit oberer linker Ecke bei \texttt{x}, \texttt{y} und Seitenlänge \texttt{s}. \\
\hline

\lstinline|ellipse(float x, float y, float w, float h)| & 
\begin{tabular}[t]{@{}l@{}}
\texttt{float x}: \textit{Zentrum X} \\
\texttt{float y}: \textit{Zentrum Y} \\
\texttt{float w}: \textit{Breite} \\
\texttt{float h}: \textit{Höhe}
\end{tabular}
& Zeichnet eine Ellipse zentriert bei \texttt{x}, \texttt{y} mit Breite \texttt{w} und Höhe \texttt{h}. \\
\hline

\lstinline|circle(float x, float y, float d)| & 
\begin{tabular}[t]{@{}l@{}}
\texttt{float x}: \textit{Zentrum X} \\
\texttt{float y}: \textit{Zentrum Y} \\
\texttt{float d}: \textit{Durchmesser}
\end{tabular}
& Zeichnet einen Kreis zentriert bei \texttt{x}, \texttt{y} mit Durchmesser \texttt{d}. \\
\hline

\lstinline|triangle(float x1, float y1, float x2, float y2, float x3, float y3)| & 
\begin{tabular}[t]{@{}l@{}}
\texttt{float x1}: \textit{Punkt1 X} \\
\texttt{float y1}: \textit{Punkt1 Y} \\
\texttt{float x2}: \textit{Punkt2 X} \\
\texttt{float y2}: \textit{Punkt2 Y} \\
\texttt{float x3}: \textit{Punkt3 X} \\
\texttt{float y3}: \textit{Punkt3 Y}
\end{tabular}
& Zeichnet ein Dreieck mit Eckpunkten bei den angegebenen Koordinaten. \\
\hline

\lstinline|quad(float x1, float y1, float x2, float y2, float x3, float y3, float x4, float y4)| & 
\begin{tabular}[t]{@{}l@{}}
\texttt{float x1}: \textit{Punkt1 X} \\
\texttt{float y1}: \textit{Punkt1 Y} \\
\texttt{float x2}: \textit{Punkt2 X} \\
\texttt{float y2}: \textit{Punkt2 Y} \\
\texttt{float x3}: \textit{Punkt3 X} \\
\texttt{float y3}: \textit{Punkt3 Y} \\
\texttt{float x4}: \textit{Punkt4 X} \\
\texttt{float y4}: \textit{Punkt4 Y}
\end{tabular}
& Zeichnet ein Viereck mit Eckpunkten bei den angegebenen Koordinaten. \\
\hline

\lstinline|arc(float x, float y, float w, float h, float start, float stop)| & 
\begin{tabular}[t]{@{}l@{}}
\texttt{float x}: \textit{Zentrum X} \\
\texttt{float y}: \textit{Zentrum Y} \\
\texttt{float w}: \textit{Breite} \\
\texttt{float h}: \textit{Höhe} \\
\texttt{float start}: \textit{Startwinkel} \\
\texttt{float stop}: \textit{Endwinkel}
\end{tabular}
& Zeichnet einen Bogen zentriert bei \texttt{x}, \texttt{y} von Winkel \texttt{start} bis \texttt{stop} (in Radiant). \\
\hline


\end{longtable}

% Wiederhole den Vorgang für die restlichen Tabellen

\end{document}
