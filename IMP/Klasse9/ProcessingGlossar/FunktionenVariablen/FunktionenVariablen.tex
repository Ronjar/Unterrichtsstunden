\documentclass{article}
\usepackage[utf8]{inputenc}
\usepackage[T1]{fontenc}
\usepackage[ngerman]{babel}
\usepackage{longtable}
\usepackage{booktabs}
\usepackage{listings}
\usepackage{array}
\usepackage{geometry}
\usepackage{hyperref}
\usepackage{parskip}
\geometry{a4paper, margin=2cm}

\usepackage{inconsolata}

\usepackage{helvet}
\renewcommand{\familydefault}{\sfdefault}


\usepackage{color}

% Solarized Farben definieren
\definecolor{solarBase03}{rgb}{0.0, 0.11, 0.16}
\definecolor{solarYellow}{rgb}{0.93, 0.71, 0.0}
\definecolor{solarRed}{rgb}{0.86, 0.26, 0.18}
\definecolor{solarGreen}{rgb}{0.18, 0.55, 0.34}
\definecolor{solarBlue}{rgb}{0.0, 0.39, 0.51}
\definecolor{solarGray}{rgb}{0.58, 0.58, 0.58}
\definecolor{solarLightGray}{rgb}{0.95, 0.95, 0.95} % Neue helle Hintergrundfarbe

% lstset für Solarized
\lstset{
  language=Java,
  backgroundcolor=\color{solarLightGray},
  basicstyle=\ttfamily\color{solarGray},
  keywordstyle=\color{solarBlue}\bfseries,
  commentstyle=\color{solarGreen}\itshape,
  stringstyle=\color{solarRed},
  showspaces=false,
  showtabs=false,
  breaklines=true,
  showstringspaces=false,
  breakatwhitespace=true,
  moredelim=[il][\textcolor{solarGray}]{$$},
  moredelim=[is][\textcolor{solarGray}]{\%\%}{\%\%},
  frame=single,
  rulecolor=\color{solarGray},
  numbers=left,
  numberstyle=\tiny\color{solarGray},
  stepnumber=1,
  numbersep=5pt
}

\begin{document}

\title{Grundlegende Variablentypen und Funktionen in Processing}
\author{}
\date{}
\maketitle

\section*{Variablentypen}

In Processing gibt es verschiedene Variablentypen, die verwendet werden, um Daten zu speichern und zu manipulieren. Jeder Typ hat seinen eigenen Verwendungszweck und Speicherbedarf.

\begin{longtable}{|p{0.3\textwidth}|p{0.3\textwidth}|p{0.35\textwidth}|}
\hline
\textbf{Typ} & \textbf{Name} & \textbf{Erklärung} \\
\hline
\endfirsthead

\hline
\textbf{Typ} & \textbf{Name} & \textbf{Erklärung} \\
\hline
\endhead

\texttt{int} & Ganzzahl & Speichert ganze Zahlen ohne Dezimalstellen. \\
\hline

\texttt{float} & Gleitkommazahl & Speichert Zahlen mit Dezimalstellen. \\
\hline

\texttt{double} & Doppelte Gleitkommazahl & Ähnlich wie \texttt{float}, jedoch mit doppelter Genauigkeit. \\
\hline

\texttt{char} & Zeichen & Speichert einzelne Zeichen, z.B. Buchstaben oder Symbole. \\
\hline

\texttt{boolean} & Wahrheitswert & Speichert \texttt{true} oder \texttt{false}. \\
\hline

\texttt{String} & Zeichenkette & Speichert Textinformationen, z.B. Wörter oder Sätze. \\
\hline

\texttt{color} & Farbe & Speichert Farbwerte, oft in RGB-Format. \\
\hline

\texttt{PVector} & Vektor & Speichert einen Vektor mit x-, y- und z-Koordinaten, nützlich für 2D- und 3D-Grafiken. \\
\hline


\end{longtable}

\section*{Variablen deklarieren und initialisieren}

Eine Variable wird \textbf{deklariert}, indem der Typ und der Name der Variable angegeben werden. Sie kann gleichzeitig \textbf{initialisiert} werden, indem ein Anfangswert zugewiesen wird.

\begin{lstlisting}
int alter = 25;
float temperatur = 23.5;
boolean istAktiv = true;
String name = "Max Mustermann";
color hintergrundFarbe = color(255, 0, 0); // Rot
\end{lstlisting}

\section*{Funktionen}

Funktionen sind Blöcke von Code, die eine bestimmte Aufgabe ausführen. Sie helfen, den Code zu strukturieren, wiederverwendbar zu machen und die Lesbarkeit zu verbessern.

\subsection*{Aufbau einer Funktion}

Eine Funktion besteht aus einem Rückgabetyp, einem Namen und optionalen Parametern. Der Funktionskörper enthält den Code, der ausgeführt wird, wenn die Funktion aufgerufen wird.

\begin{lstlisting}
Rueckgabetyp Funktionsname(Parameterliste) {
    // Funktionskoerper
    // Code, der ausgefuehrt wird
    return Wert; // Falls ein Wert zurueckgegeben wird
}
\end{lstlisting}

\subsection*{Beispiel einer Funktion}

Hier ist ein einfaches Beispiel einer Funktion, die zwei Ganzzahlen addiert und das Ergebnis zurückgibt.

\begin{lstlisting}
int addiere(int a, int b) {
    int summe = a + b;
    return summe;
}
\end{lstlisting}

\subsection*{Funktionen aufrufen}

Eine Funktion wird aufgerufen, indem ihr Name gefolgt von Klammern und den erforderlichen Argumenten verwendet wird.

\begin{lstlisting}
int ergebnis = addiere(5, 3); // ergebnis ist jetzt 8
\end{lstlisting}

\section*{Verschiedene Arten von Funktionen}

\begin{longtable}{|p{0.25\textwidth}|p{0.25\textwidth}|p{0.35\textwidth}|}
\hline
\textbf{Art} & \textbf{Name} & \textbf{Erklärung} \\
\hline
\endfirsthead

\hline
\textbf{Art} & \textbf{Name} & \textbf{Erklärung} \\
\hline
\endhead

\texttt{void} & Prozedur & Führt Aufgaben aus, gibt jedoch keinen Wert zurück. \\
\hline

\texttt{int}, \texttt{float}, etc. & Funktion mit Rückgabewert & Führt Aufgaben aus und gibt einen Wert des angegebenen Typs zurück. \\
\hline

\end{longtable}

\section*{Eingebaute Funktionen in Processing}

\begin{longtable}{|p{0.25\textwidth}|p{0.25\textwidth}|p{0.35\textwidth}|}
    \hline
    \textbf{Art} & \textbf{Name} & \textbf{Erklärung} \\
    \hline
    \endfirsthead

\texttt{void setup()} & Setup-Funktion & Wird einmal zu Beginn des Programms ausgeführt, um Einstellungen vorzunehmen. \\
\hline

\texttt{void draw()} & Draw-Funktion & Wird kontinuierlich nach \texttt{setup()} aufgerufen, um Grafiken zu zeichnen oder Animationen zu erstellen. \\
\hline


\end{longtable}

\section*{Parameter und Argumente}

\begin{itemize}
    \item \textbf{Parameter}: Variablen, die in der Funktionsdefinition angegeben sind und als Platzhalter für die Werte dienen, die an die Funktion übergeben werden.
    \item \textbf{Argumente}: Die tatsächlichen Werte, die beim Aufruf der Funktion übergeben werden.
\end{itemize}

\subsection*{Beispiel mit Parametern und Argumenten}

\begin{lstlisting}
float berechneFlaeche(float breite, float hoehe) {
    return breite * hoehe;
}

void setup() {
    float flaeche = berechneFlaeche(5.0, 3.0); // flaeche ist jetzt 15.0
}
\end{lstlisting}

\section*{Lokale und Globale Variablen}

\begin{itemize}
    \item \textbf{Lokale Variablen}: Innerhalb einer Funktion definiert und nur innerhalb dieser Funktion sichtbar.
    \item \textbf{Globale Variablen}: Außerhalb aller Funktionen definiert und in allen Funktionen des Programms zugänglich.
\end{itemize}

\subsection*{Beispiel für lokale und globale Variablen}

\begin{lstlisting}
int globalZahl = 10; // Globale Variable

void setup() {
    int lokaleZahl = 5; // Lokale Variable
    println("Globale Zahl: " + globalZahl);
    println("Lokale Zahl: " + lokaleZahl);
}

void draw() {
    println("Globale Zahl in draw: " + globalZahl);
    // println("Lokale Zahl in draw: " + lokaleZahl); // Fehler: lokaleZahl ist hier nicht sichtbar
}
\end{lstlisting}

\end{document}
